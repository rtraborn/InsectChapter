
%%%%%%%%%%%%%%%%%%%%%%% file typeinst.tex %%%%%%%%%%%%%%%%%%%%%%%%%
%
% This is the LaTeX source for the instructions to authors using
% the LaTeX document class 'llncs.cls' for contributions to
% the Lecture Notes in Computer Sciences series.
% http://www.springer.com/lncs       Springer Heidelberg 2006/05/04
%
% It may be used as a template for your own input - copy it
% to a new file with a new name and use it as the basis
% for your article.
%
% NB: the document class 'llncs' has its own and detailed documentation, see
% ftp://ftp.springer.de/data/pubftp/pub/tex/latex/llncs/latex2e/llncsdoc.pdf
% Source URL: https://www.overleaf.com/9872043pgfkbjhyncvj#/36134551/
%%%%%%%%%%%%%%%%%%%%%%%%%%%%%%%%%%%%%%%%%%%%%%%%%%%%%%%%%%%%%%%%%%%


\documentclass[runningheads,a4paper]{llncs}

\usepackage{amssymb}
\setcounter{tocdepth}{3}
\usepackage{graphicx}
\usepackage{authblk}
\usepackage{lineno}
\usepackage{listings}

%\urldef{\mailsa}\path|{rtraborn, |
%\urldef{\mailsb}\path|vbrendel}@indiana.edu|    

\newcommand{\keywords}[1]{\par\addvspace\baselineskip
\noindent\keywordname\enspace\ignorespaces#1}

\begin{document}

\mainmatter  % start of an individual contribution

% first the title is needed
\title{Using RAMPAGE to identify and annotate regulatory elements in insect genomes}

% a short form should be given in case it is too long for the running head
\titlerunning{Identification of insect regulatory elements}

% the name(s) of the author(s) follow(s) next
%
% NB: Chinese authors should write their first names(s) in front of
% their surnames. This ensures that the names appear correctly in
% the running heads and the author index.
%
\author[1,2]{R. Taylor Raborn\thanks{Correspondence: rtraborn@indiana.edu}}
\author[1,2]{Volker P. Brendel}

\affil[1]{Department of Biology, Indiana University}
\affil[2]{School of Informatics and Computing, Indiana University}

%\date{\today}

\renewcommand\Authands{ and }
%
\authorrunning{Raborn and Brendel}
% (feature abused for this document to repeat the title also on left hand pages)

% the affiliations are given next; don't give your e-mail address
% unless you accept that it will be published
\institute{Department of Biology and School of Informatics and Computing, \\
Indiana University\\
212 S. Hawthorne Drive 205 Simon Hall, Bloomington, IN 47401, USA\\
%\mailsa\\
%\mailsb\\
\url{http://www.brendelgroup.org}}

%
% NB: a more complex sample for affiliations and the mapping to the
% corresponding authors can be found in the file "llncs.dem"
% (search for the string "\mainmatter" where a contribution starts).
% "llncs.dem" accompanies the document class "llncs.cls".
%

\toctitle{Lecture Notes in Computer Science}
\tocauthor{Authors' Instructions}
\maketitle


\begin{abstract}
Application of Transcription Start Site (TSS) profiling technologies, coupled with large-scale next-generation sequencing (NGS) has yielded valuable insights into the location, structure and activity of promoters across diverse metazoan model systems.
In insects, TSS profiling has been used to characterize the promoter architecture of \textit{D. melanogaster}, and, shortly thereafter, to reveal widespread transposon-driven alternative promoter usage. 

In this chapter we highlight the utility of one TSS profiling method, RAMPAGE (RNA annotation and mapping of promoters for analysis of gene expression), for the precise, quantitative identification of promoters in insect genomes.
We demonstrate this using our tools GoRAMPAGE and TSRchitect, providing details instructions with the aim of taking the user from raw reads to processed results. 

\keywords{\textit{cis}-regulatory regions, promoter architecture, transcription initiation, transcription start sites (TSSs)}
\end{abstract}

\begin{linenumbers}
\section{Introduction}

\subsection{TSS Profiling Identifies Promoters at Genome-Scale}
The promoter, defined in eukaryotes as the genomic region bound by RNA Polymerase II immediately prior to transcription initiation \cite{Kadonaga:2011gz}, is the site where regulatory signals unite to direct gene expression.
The identification of promoter regions is a valuable step for understanding the \textit{cis}-regulatory signals that are present in an organism, and is important for genome annotation.
However, despite the rapid accumulation of genome sequences across metazoan and arthropod diversity, accurate annotation of promoter regions remains sparse. 
This is because---empirical mapping of TSSs---precisely identifying sequence motifs that demarcate the promoter is unreliable.
In contrast with current \textit{in silico} approaches, direct mapping of TSSs identifies the location of the core promoter.
Cap Analysis of Gene Expression (CAGE) \cite{Kodzius:2006gy}, one of the first methods devised to identify 5`-ends of mRNAs at large-scale, involves selective capture of 5`-capped transcripts, first-strand reverse-transcription and ligation of a short oligonucleotide (CAGE tag). 
CAGE was initially utilized by the FANTOM (Functional Annotation of the Mammalian Genome) consortium to identify promoter architecture in human and mouse \cite{Carninci:2005kp}, providing the first glimpse of the global landscape of transcription initiation.
At the onset of the NGS era, CAGE was coupled with massively-parallel sequencing to generate 5`-ends of mRNAs at substantially higher scale. 
This advance provided more extensive coverage of the expressed transcriptome, and provided increased sensitivity for quantitative measurements \textit{i}.\textit{e}. measurement of promoter activity.

\subsection{Promoter Architecture of \textit{Drosophila melanogaster}}
Hoskins and colleagues \cite{Hoskins:2011io} performed CAGE in \textit{D. melanogaster} as part of the modENCODE consortium, identifying promoters at large-scale and characterizing the promoter architecture of an insect genome for the first time.
Hoskins \cite{Hoskins:2011io} indicated that TSS distributions at \textit{Drosophila} promoters exhibit a range of shapes that can be generally grouped into two major classifications: \textit{peaked} and \textit{broad}. 
Peaked promoters have a single, major TSS position occupying a narrow genomic region, whereas broad promoters lack a single, major TSS and contain TSSs across a wider region \cite{Rach:2009ct}\cite{Lenhard:2012en}. 
The authors also showed a strong association between promoter class and motif composition (consistent with previous findings \cite{Rach:2009ct,Ni:2010jh}). 
Peaked promoters were associated with positionally-enriched \textit{cis}-regulatory motifs including TATA, Initiator (Inr) and DPE, while broad promoters contained an enrichment of less-well characterized motifs, including \textit{Ohler6} and \textit{Ohler7} \cite{Ohler:2002vl}. 
The existence of two promoter classes appears to be conserved among metazoans, and has been reported (using TSS profiling methodolgies) in insects, cladocerans \cite{Raborn:2016cr}, fish \cite{Nepal:2013bga} and mammals \cite{Carninci:2006in,Lenhard:2012en}.

\subsection{Promoter Structure of Insects}
Beyond \textit{D. melanogaster}, few investigations have utilized TSS profiling in insect genomes. 
As a consequence, what is known about promoter architecture in insects is largely restricted to the \textit{Drosophila} genus. 
As part of the modENCODE effort, CAGE was performed in multiple tissues and developmental stages of the \textit{Drosophila pseudoobscura}. 
TSSs were found to be highly similar between species: more than 80\% of TSSs (81\%) of aligned, CAGE-identified TSSs from \textit{D. pseudoobscura} were positioned within 20nt of their counterparts in \textit{D. melanogaster}.
An enrichment of the CA dinucleotide was detected at the TSS ([-1, +1]), and the motifs corresponding to TATA, Inr and DPE were positioned at the same locations relative to the TSS in both species.
The one other insect species for which TSS profiling has been applied is the Tsetse fly (\textit{Glossina morsitans morsitans}) \cite{Mwangi:2015kn}. 
Using TSS-seq (specifically Oligo-capping; for details on this method see \cite{Tsuchihara:2009dm}), the authors identified 3134 mapping to 1424 genes. 
The authors found a preference for CA and AA dinucleotides at the TSS, and observe the major core promoter elements observed in \textit{Drosophila}: TATA, Inr, DPE, in addition to MTE (Motif Ten Element).
As in \textit{D. melanogaster}, peaked promoters were more likely to contain TATA and Inr than broad promoters. 
While the taxonomic sampling of species for TSS profiling has been limited, the existing studies are sufficient to provide a general picture of insect promoter architecture.
A major demarcation between the promoter architecture of insects and mammals appears to be the large fraction of mammalian promoters found in CpG islands \cite{Mwangi:2015kn}.
CpG island promoters (CPIs) form the largest class of promoter in mammals \cite{Cvetesic:2017hl}; by contrast, CPIs are not known to exist as a class in invertebrates.


\subsection{Paired-end TSS Profiling with RAMPAGE}
The most recent major methodological advance in TSS Profiling is RAMPAGE (RNA Annotation and Mapping of Promoters for the Analysis of Gene Expression)  . 
RAMPAGE is a protocol for 5'-cDNA sequencing that combines cap trapping and template-switching with paired-end sequence information. 
A key advantage of generating paired-end sequence is transcript connectivity, which provides a direct link between a given 5'-end and its associated mRNA molecule.
Because short or spurious RNAs are found within the transcriptome, transcript connectivity allows the TSSs (and thus promoters) of full-length mRNAs to be unambiguously identified, which benefits genome annotation.
Batut and colleagues generated libraries from total RNA isolated from 36 stages across the life cycle of \textit{D. melanogaster} providing a comprehensive gene expression and promoter atlas for fruit fly and in the process demonstrating the utility of RAMPAGE.
RAMPAGE is currently being applied as part of the latest iteration of ENCODE to identify promoters in human, but as of this writing it has not been applied to any non-\textit{Drosophila} insect species. 
In anticipation of the future application of TSS profiling into other insect model systems here we provide a documented protocol for the computational processing RAMPAGE data, using selected libraries from Batut \textit{et al.}. 
This method will consist of two parts: first, we will process, filter and align the sequenced RAMPAGE libraries to the \textit{D. melanogaster} genome. 
Second, we will identify TSSs and promoters from the aligned sequences and associate them with coding regions.
In closing, we will consider further applications of this data and discuss the utility of reproducible workflows in bioinformatic analysis.

%You are strongly encouraged to use \LaTeXe{} for the
%preparation of your camera-ready manuscript together with the
%corresponding Springer class file \verb+llncs.cls+. Only if you use
%\LaTeXe{} can hyperlinks be generated in the online version
%of your manuscript.

\section{Materials}

The analyses described herein require a workstation capable for modern bioinformatics. 
An intermediate understanding of the Linux/Unix command line will be extremely useful, although we make efforts to explain the procedures with clarity. In addition, it will likely be necessary for the participant to have superuser privileges on the machine.
If you do not have a machine (or access to one) that meets these requirements, it is recommended that you consider cloud-based cyberinfrastructure, including Amazon Web Services (AWS; https://aws.amazon.com/) or CyVerse (http://www.cyverse.org/).
The former is a well-known pay-per-use solution, while the latter is an NSF-funded resource that is made freely available to the public.

\subsection{Hardware Requirements}
\begin{itemize} 
\item x86-64 compatible processors
\item At least 8GB RAM
\item 30GB+ hard disk space
\end{itemize}

\subsection{Software Requirements}
\begin{itemize}
\item Operating 64 bit Linux (preferred) or Mac OS X (with Command Line Tools from XCode)
\item R (version 3.4) 
\item Bioconductor (version 3.5)
\item FASTX-Toolit (version 0.0.13)
\item Samtools (version 1.3 or above)
\item SRA Toolkit (version 2.3.4-2 or above)
\item STAR aligner (version 2.4 or above)
\item TagDust (version 2.33)

\subsection{Installation of R packages}
For installation of the software listed above, please follow the instructions provided by each respective package. 
Part of our analysis will require the use of R packages found in the Bioconductor suite.
To install Bioconductor, please type the following from an R console: 

\begin{lstlisting}
source("https://bioconductor.org/biocLite.R")
biocLite()
\end{lstlisting}

We will use the R package \textit{TSRchitect} to identify promoters from aligned RAMAPGE libraries. 
First, we will need to install a series of prerequisite packages to \textit{TSRchitect} from Bioconductor.
Please install these packages as follows (as before, from an R console):


\begin{lstlisting}
source("https://bioconductor.org/biocLite.R")
biocLite(c("AnnotationHub", "BiocGenerics", "BiocParallel", "ENCODExplorer", \
 "GenomicAlignments", "GenomeInfoDb", "GenomicRanges", "IRanges", "methods", \
 "Rsamtools", "rtracklayer", "S4Vectors", "SummarizedExperiment"))
\end{lstlisting}

To install \textit{TSRchitect}, please type the following from an R console:

\begin{lstlisting}
source("https://bioconductor.org/biocLite.R")
biocLite("TSRchitect")
\end{lstlisting}

Finally, please confirm that TSRchitect has been installed correctly by loading it from your R console as follows:

\begin{lstlisting}
library(TSRchitect)
\end{lstlisting}

\section{Methods}


\section{Notes}

\subsection*{Acknowledgments}

\subsection*{Disclosure Declaration}
The authors declare that they have no competing interests.
\end{linenumbers}

\section{Figures}

For \LaTeX\ users, we recommend using the \emph{graphics} or \emph{graphicx}
package and the \verb+\includegraphics+ command.

Please check that the lines in line drawings are not
interrupted and are of a constant width. Grids and details within the
figures must be clearly legible and may not be written one on top of
the other. Line drawings should have a resolution of at least 800 dpi
(preferably 1200 dpi). The lettering in figures should have a height of
2~mm (10-point type). Figures should be numbered and should have a
caption which should always be positioned \emph{under} the figures, in
contrast to the caption belonging to a table, which should always appear
\emph{above} the table; this is simply achieved as matter of sequence in
your source.

Please center the figures or your tabular material by using the \verb+\centering+
declaration. Short captions are centered by default between the margins
and typeset in 9-point type (Fig.~\ref{fig:example} shows an example).
The distance between text and figure is preset to be about 8~mm, the
distance between figure and caption about 6~mm.

To ensure that the reproduction of your illustrations is of a reasonable
quality, we advise against the use of shading. The contrast should be as
pronounced as possible.

If screenshots are necessary, please make sure that you are happy with
the print quality before you send the files.
\begin{figure}
\centering
\includegraphics[height=6.2cm]{eijkel2}
\caption{One kernel at $x_s$ (\emph{dotted kernel}) or two kernels at
$x_i$ and $x_j$ (\textit{left and right}) lead to the same summed estimate
at $x_s$. This shows a figure consisting of different types of
lines. Elements of the figure described in the caption should be set in
italics, in parentheses, as shown in this sample caption.}
\label{fig:example}
\end{figure}

Please define figures (and tables) as floating objects. Please avoid
using optional location parameters like ``\verb+[h]+" for ``here".

\subsection{Formulas}

Displayed equations or formulas are centered and set on a separate
line (with an extra line or halfline space above and below). Displayed
expressions should be numbered for reference. The numbers should be
consecutive within each section or within the contribution,
with numbers enclosed in parentheses and set on the right margin --
which is the default if you use the \emph{equation} environment, e.g.,
\begin{equation}
  \psi (u) = \int_{o}^{T} \left[\frac{1}{2}
  \left(\Lambda_{o}^{-1} u,u\right) + N^{\ast} (-u)\right] dt \;  .
\end{equation}

Equations should be punctuated in the same way as ordinary
text but with a small space before the end punctuation mark.

\subsection{Footnotes}

The superscript numeral used to refer to a footnote appears in the text
either directly after the word to be discussed or -- in relation to a
phrase or a sentence -- following the punctuation sign (comma,
semicolon, or period). Footnotes should appear at the bottom of
the
normal text area, with a line of about 2~cm set
immediately above them.\footnote{The footnote numeral is set flush left
and the text follows with the usual word spacing.}

\subsection{Program Code}

Program listings or program commands in the text are normally set in
typewriter font, e.g., CMTT10 or Courier.

\medskip

\noindent
{\it Example of a Computer Program}
\begin{verbatim}
program Inflation (Output)
  {Assuming annual inflation rates of 7%, 8%, and 10%,...
   years};
   const
     MaxYears = 10;
   var
     Year: 0..MaxYears;
     Factor1, Factor2, Factor3: Real;
   begin
     Year := 0;
     Factor1 := 1.0; Factor2 := 1.0; Factor3 := 1.0;
     WriteLn('Year  7% 8% 10%'); WriteLn;
     repeat
       Year := Year + 1;
       Factor1 := Factor1 * 1.07;
       Factor2 := Factor2 * 1.08;
       Factor3 := Factor3 * 1.10;
       WriteLn(Year:5,Factor1:7:3,Factor2:7:3,Factor3:7:3)
     until Year = MaxYears
end.
\end{verbatim}
%
\noindent
{\small (Example from Jensen K., Wirth N. (1991) Pascal user manual and
report. Springer, New York)}

\subsection{Citations}

For citations in the text please use
square brackets and consecutive numbers: \cite{jour}, \cite{lncschap},
\cite{proceeding1} -- provided automatically
by \LaTeX 's \verb|\cite| \dots\verb|\bibitem| mechanism.

\subsection{Page Numbering and Running Heads}

There is no need to include page numbers. If your paper title is too
long to serve as a running head, it will be shortened. Your suggestion
as to how to shorten it would be most welcome.

\section{References}\label{references}

\newpage

% if your bibliography is in bibtex format, use those commands:
\bibliographystyle{IEEEtran} % Bibliography style file
\bibliography{RAMPAGE_chapter}      % Bibliography file

In order to permit cross referencing within LNCS-Online, and eventually
between different publishers and their online databases, LNCS will,
from now on, be standardizing the format of the references. This new
feature will increase the visibility of publications and facilitate
academic research considerably. Please base your references on the
examples below. References that don't adhere to this style will be
reformatted by Springer. You should therefore check your references
thoroughly when you receive the final pdf of your paper.
The reference section must be complete. You may not omit references.
Instructions as to where to find a fuller version of the references are
not permissible.

We only accept references written using the latin alphabet. If the title
of the book you are referring to is in Russian or Chinese, then please write
(in Russian) or (in Chinese) at the end of the transcript or translation
of the title.

The following section shows a sample reference list with entries for
journal articles \cite{jour}, an LNCS chapter \cite{lncschap}, a book
\cite{book}, proceedings without editors \cite{proceeding1} and
\cite{proceeding2}, as well as a URL \cite{url}.
Please note that proceedings published in LNCS are not cited with their
full titles, but with their acronyms!

\section{Checklist of Items to be Sent to Volume Editors}
Here is a checklist of everything the volume editor requires from you:

\begin{itemize}
\settowidth{\leftmargin}{{\Large$\square$}}\advance\leftmargin\labelsep
\itemsep8pt\relax
\renewcommand\labelitemi{{\lower1.5pt\hbox{\Large$\square$}}}

\item The final \LaTeX{} source files
\item A final PDF file
\item A copyright form, signed by one author on behalf of all of the
authors of the paper.
\item A readme giving the name and email address of the
corresponding author.
\end{itemize}
\end{document}
